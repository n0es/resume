%%%%%%%%%%%%%%%%%%%%%%%%%%%%%%%%%%%%%%%%%%%%%%%%%%%%%%%%%%%%%%%%%%%
%% 
%% Ethan White's resume
%%   - based off work by Michael DeCorte and Yisong Yue
%%
%%%%%%%%%%%%%%%%%%%%%%%%%%%%%%%%%%%%%%%%%%%%%%%%%%%%%%%%%%%%%%%%%%%



%%
%% The following code sets up the document formatting
%%

%this assumes that res_yy.sty is in some path
\documentstyle[hyperref, margin, line]{res_yy}

\hypersetup{backref,pdfpagemode=Full,colorlinks=true,backref}

\addtolength{\oddsidemargin}{-0.45in}
\addtolength{\voffset}{-0.30in}
\addtolength{\textwidth}{1.00in} \addtolength{\textheight}{1.50in}

\renewcommand{\namefont}{\LARGE\emph}



%%
%% The following code defines some macros for terms which have raised font
%% (ie 4\fourth would result 4th with the 'th' raised (superscripted)
%%

\def\Cplusplus{{\rm C\raise.5ex\hbox{\small ++}}}
\def\CSharp{{\rm C\raise.5ex\hbox{\small \#}}}
% 'st' 'nd' 'rd' 'th' superscripts for numbers
\def\first{{\raise.5ex\hbox{\small st}}}
\def\second{{\raise.5ex\hbox{\small nd}}}
\def\third{{\raise.5ex\hbox{\small rd}}}
\def\fourth{{\raise.5ex\hbox{\small th}}}



%%
%% starting the actual document
%%

\begin{document}

%the name in big fonts at the top of resume
%this is left aligned
\name{Ethan White}

%this is right aligned
\address{
website: www.boopurno.es \ \ \ \ \ email: ethanwhi@gmail.com  
}

\begin{resume}



%%
%% This section of code is inelegant, but I'm too lazy to fix it
%%

% \section{\textsc{Research Interest}}
% My research interests lie primarily in machine learning, data analysis and information retrieval. 

\section{\textsc{Education}}

\textbf{Seton Hall University} \ \ \ \ \ \ \ \ \ \ \ \ \ \ \ \ \ \ \ \ \ \ \ \ \ \ \ \ \ \ \ \ \ \ \ \ \ \ \ \ \ \ \ \ \ \ \ \ \ \ \ \ \ \ \ \ \ \ \ \ \ \ \ \ \ \ \ \ \ \ \ \ \ \ \ \ \ \ \ 2023 - current \\
Bachelor of Science in Computer Science

\textbf{Scotch Plains Fanwood High School} \ \ \ \ \ \ \ \ \ \ \ \ \ \ \ \ \ \ \ \ \ \ \ \ \ \ \ \ \ \ \ \ \ \ \ \ \ \ \ \ \ \ \ \ \ \ \ \ \ \ \ \ \ \ \ 2019 - 2023 \\ 
High School Diploma



%%
%% the meat of the resume starts now
%%

\begin{formatb}
  \employer{l}\title{r}\\
  \location{l}\dates{r}\\
  \body\\
\end{formatb}

\section{\textsc{Work Experience}}

\employer{\textbf{Upperlevel Hosting}}
\title{Junior Web Developer}
\location{Remote}
\dates{2022 -- 2023}
\begin{position}
Automated tasks using php and bash, and wrote unit tests for existing code.
\end{position}


%%
%% We use the same formatting for projects as for work experience
%% Shown below is the formatting used previously
%%
%%  \begin{formatb}
%%    \employer{l}\title{r}\\
%%    \location{l}\dates{r}\\
%%    \body\\
%%  \end{formatb}
%%
%% 
%%  Note that \location is now being used for non-location information
%%


\begin{formatb}
  \employer{l}\dates{r}\\
  \body\\
\end{formatb}

\section{\textsc{Projects}}

\employer{\textbf{SHULIB}}
\dates{Fall 2024}
\begin{position}
Library of functions and classes used by the Seton Hall University Robotics Team.
\end{position}

\employer{\textbf{SHULIB Dashboard}}
\dates{Fall 2024}
\begin{position}
A dashboard program developed using python and tkinter to visualize data from the SHULIB library.
\end{position}

\employer{\textbf{Moo Bot}}
\dates{Spring 2022}
\begin{position}
A modular discord bot developed using the MERN stack to provide useful features to any discord server.
\end{position}

%%
%% This section could also use more formatting, but looks ok, as is
%%

%%
%% Note that we're redefining the formatting
%% We only have one row of information now, instead of two
%%

\section{\textsc{Activities}}

\begin{formatb}
  \employer{l}\dates{r}\\
  \body\\
\end{formatb}

\employer{\textbf{Seton Hall University Robotics Lead Programmer}}
\dates{Fall 2023 - Current}
\begin{position}
Lead programmer for the Seton Hall University Robotics Team, responsible for creating the SHULIB library and other tools to support the team.
\end{position}

\employer{\textbf{SHU ACM SIGPLAN Member}}
\dates{Fall 2024 - Current}
\begin{position}
Member of the Seton Hall University ACM SIGPLAN chapter.
\end{position}

\section{\textsc{Qualifications}}

\emph{Programming Languages}: \Cplusplus, PHP, Java, Python, JavaScript, TypeScript, React

\emph{Libraries and Tools}: \LaTeX, Adobe Suite, Microsoft Visual Studio, Git, Docker



%%
%% Nothing special here, just a normal table
%%

%\section{\textsc{Course Work}}
%  \begin{tabular}{lllll}
%  Information Networks   & \ \ & Machine Learning    & \ \ & Theory of Computation \\ 
%  Computer Graphics      & \ \ & Machine Vision      & \ \ & Programming Languages \\
%  Software Engineering   & \ \ & Algorithms          & \ \ & Artificial Intelligence     \\
%  Operating Systems      & \ \ & Databases           & \ \ & Computer Architecture \\
%  Numerical Methods      & \ \ & Graph Theory        & \ \ & Differential Equations      \\
%  Probability Theory     & \ \ & Number Theory       & \ \ & Differential Geometry       \\
%  Advanced Calculus      & \ \ & Abstract Algebra    & \ \ & Advanced Combinatorics   \\
%  \end{tabular}


\end{resume}
\end{document}